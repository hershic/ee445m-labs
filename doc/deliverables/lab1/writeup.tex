\documentclass[12pt]{article}
\title{EE445M Lab 1}
\author{Hershal Bhave (hb6279) and Eric Crosson (esc625)}
\date{Due Sometime Soon}

\usepackage[in]{fullpage}
\usepackage{listings}
\usepackage{cleveref}
\usepackage[nosolutionfiles]{answers}
\usepackage{xcolor}
\usepackage{color}
\usepackage{enumerate}

\newenvironment{Ex}{\textbf{Problem}\vspace{.75em}\\}{}
\Newassociation{solution}{Soln}{Answers}
\pagebreak[3]
\newcommand{\Opentesthook}[2]{\Writetofile{#1}{\protect\section{#1: #2}}}
\renewcommand{\Solnlabel}[1]{\textbf{Solution}\quad}
\newcommand{\todo}{{\LARGE \emph{\color{red}TODO}}}

\newcommand{\dd}[1]{\:\mathrm{d}{#1}}
\newcommand{\ddt}[1]{\frac{\dd{}}{\dd{#1}}}
\newcommand{\dddt}[1]{\frac{\dd{}^2}{\dd{#1}^2}}

\definecolor{mygreen}{rgb}{0,0.6,0}
% \definecolor{mygreen}{rgb}{0.13,0.55,0.13}
\definecolor{mygray}{rgb}{0.5,0.5,0.5}
\definecolor{mymauve}{rgb}{0.58,0,0.82}

\lstset{
  backgroundcolor=\color{white},
  basicstyle=\scriptsize\ttfamily,
  breakatwhitespace=false,
  breaklines=true,
  captionpos=b,
  commentstyle=\color{mygreen},
  deletekeywords={...},
  escapeinside={\%*}{*)},
  extendedchars=true,
  frame=single,
  keywordstyle=\color{blue},
  % language=Octave,
  % numbers=left,
  % numbersep=5pt,
  % numberstyle=\tiny\color{mygray},
  rulecolor=\color{black},
  showspaces=false,
  showstringspaces=false,
  showtabs=false,
  % stepnumber=2,
  stringstyle=\color{mymauve},
  tabsize=2,
  title=\lstname,
  columns=fullflexible,
}

\begin{document}
\maketitle
\section{Prepreparation}
\begin{enumerate}
\setcounter{enumi}{-1}
\item
  \begin{Ex}
    Please review the style guideline presented in style.pdf and
    \verb|c_and_h_files.pdf|. Go to the ARM site to download the
    compiler to your laptop
    \verb|https://www.keil.com/demo/eval/armv4.htm|. Do not get the
    newest compiler (Version 5.x) because it does not support the TI
    boards. Any version 4.7 to 4.73 will be OK. Download and install
    the Keil’s uVision4 compiler using these instructions
    \verb|http://www.ece.utexas.edu/~valvano/edX/KeilInstall.html|.
    \begin{solution} \hfill \vspace{.75em} \\
      Ok.
    \end{solution}
  \end{Ex}
\item
  \begin{Ex}
    Search through the \verb|UARTInts_4C123| project to answer these
    questions about the UART port.
    \begin{enumerate}
    \item This example used UART0. What lines of C code define which
      port will be used for the UART channel?
    \item What lines of C code define the baud rate, parity, data bits
      and stop bits for the UART?
    \item Which port pins are used for the UART? Which pin transmits
      and which pin receives?
    \item Look in the uart.c driver to find what low-level C code
      inputs one byte from the UART port.
    \item Similarly, find the low-level C code that outputs one byte
      to the UART port.
    \item Find in the project the interrupt vector table. In
      particular, how does the system set the ISR vector?
    \item This code \verb|UART0_ICR_R = UART_ICR_TXIC;| acknowledges a
      serial transmit interrupt. Explain how the acknowledgement
      occurs in general for all devices and in specific for this
      device.
    \item Look in the data sheet of the TM4C123 and determine the
      extent of hardware buffering of the UART channel. For example,
      the 9S12 transmitter has a transmit data register and a transmit
      shift register. So, the software can output two bytes before
      having to wait. The serial ports on the PC have 16 bytes of
      buffering. So, the software can output 16 bytes before having to
      wait. The 9S12 receiver has a receive data register and a
      receive shift register. This means the software must read the
      received data within 10 bit times after the RDRF flag is set in
      order to prevent overrun. Is the TM4C123 like the 9S12 (allowing
      just two bytes), or is it like the PC (having a larger hardware
      fifo buffer)?
    \end{enumerate}
    \begin{solution} \hfill \vspace{.75em} \\
      \begin{enumerate}
      \item \hfill
        \begin{lstlisting}[language=C]
          // enable alt funct on PA1-0
          GPIO_PORTA_AFSEL_R |= 0x03;

          // enable digital I/O on PA1-0
          GPIO_PORTA_DEN_R |= 0x03;

          // configure PA1-0 as UART
          GPIO_PORTA_PCTL_R = (GPIO_PORTA_PCTL_R&0xFFFFFF00)+0x00000011;

          // disable analog functionality on PA
          GPIO_PORTA_AMSEL_R = 0;
        \end{lstlisting}
      \item \hfill
        \begin{lstlisting}[language=C]
          // IBRD = int(50,000,000 / (16 * 115,200)) = int(27.1267)
          UART0_IBRD_R = 27;
          // FBRD = int(0.1267 * 64 + 0.5) = 8
          // 8 bit word length (no parity bits, one stop bit, FIFOs)
          UART0_FBRD_R = 8;
        \end{lstlisting}

      \item \verb|PA0| and \verb|PA1| are used for the
        UART. \verb|PA0| receives and \verb|PA1| transmits.
      \item \hfill
        \begin{lstlisting}[language=C]

          // input ASCII character from UART
          // spin if RxFifo is empty
          char UART_InChar(void){
            char letter;
            while(RxFifo_Get(&letter) == FIFOFAIL){};
            return(letter);
          }
        \end{lstlisting}
      \item \hfill
        \begin{lstlisting}[language=C]

          // output ASCII character to UART
          // spin if TxFifo is full
          void UART_OutChar(char data){
            while(TxFifo_Put(data) == FIFOFAIL){};
            // disable TX FIFO interrupt
            UART0_IM_R &= ~UART_IM_TXIM;
            copySoftwareToHardware();
            // enable TX FIFO interrupt
            UART0_IM_R |= UART_IM_TXIM;
          }
        \end{lstlisting}

      \item The vector table exists in \verb|startup.s|. The system
        finds the named function and calls that function when the
        interrupt is received.
      \item The acknowledgement occurs by setting a bit in a
        memory-mapped hardware register. For this device, bit 6 is set
        to acknowledge the interrupt.
      \item The TM4C is like a PC. It has two 16-byte FIFOs (one for
        transmit, one for receive).
      \end{enumerate}
    \end{solution}
  \end{Ex}
\item
  \begin{Ex}
    Search through the \verb|ST7735_4C123.zip| project to answer these
    questions about the LCD interface.
    \begin{enumerate}
    \item What synchronization method is used for the low-level
      command writedata?
    \item Explain the parameters of the function
      \verb|ST7735_DrawChar|. I.e., how do you use this function?
    \item Which port pins are used for the LCD? Find the connection
      diagram needed to interface the LCD.
    \item Specify which other device shares pins with the LCD.
    \end{enumerate}
    \begin{solution} \hfill \vspace{.75em} \\
      \begin{enumerate}
      \item Spinlock.
      \item Draw a character \verb|c|, at position (\verb|x|,
        \verb|y|), with color \verb|color|, background \verb|bg|, and
        font size \verb|size|.
      \item
        \begin{itemize}
        \item pin 10 (Backlight) connected to +3.3 V
        \item pin 9 (\verb|MISO|) unconnected
        \item pin 8 (\verb|SCK|) connected to PA2 (\verb|SSI0Clk|)
        \item pin 7 (\verb|MOSI|) connected to PA5 (\verb|SSI0Tx|)
        \item pin 6 (\verb|TFT_CS|) connected to PA3 (\verb|SSI0Fss|)
        \item pin 5 (\verb|CARD_CS|) unconnected
        \item pin 4 (Data/Command) connected to PA6 (\verb|GPIO|)
        \item pin 3 (\verb|RESET|) connected to PA7 (\verb|GPIO|)
        \item pin 2 (\verb|VCC|) connected to +3.3 V
        \item pin 1 (\verb|GND|) connected to ground
        \end{itemize}
      \item The I$^2$C shares PA6 and PA7. SSI pins also shared.
      \end{enumerate}
    \end{solution}
  \end{Ex}
\item
  \begin{Ex}
    Search through the \verb|PeriodicSysTickInts_4C123.zip|,
    \verb|ST7735_4C123|, and \verb|GPIO_4C123.zip| projects to answer
    these questions about the SysTick interrupts.
    \begin{enumerate}
    \item What C code defines the period of the SysTick
      interrupt?
    \item The \verb|GPIO_4C123| project runs at 16 MHz, the
      \verb|PeriodicSysTickInts_4C123| project runs at 50 MHz, and the
      \verb|ST7735_4C123| project runs at 80 MHz. Find the RCC and RCC2
      registers in the data sheet. Look at these three projects to
      explain how the system clock is established. We will be running
      at 80 MHz for most labs in the class.
    \item Look up in the data sheet what condition causes this SysTick
      interrupt and how this interrupt is acknowledged?
    \end{enumerate}
    \begin{solution}
      \begin{enumerate} \hfill \vspace{.75em} \\
      \item \verb|NVIC_ST_RELOAD_R = NVIC_ST_RELOAD_M;|
      \item The system clock is established by the \verb|PLL_Init|
        function.

        \begin{lstlisting}[language=C]
          void PLL_Init(void){
            // 0) configure the system to use RCC2 for advanced features
            //    such as 400 MHz PLL and non-integer System Clock Divisor
            SYSCTL_RCC2_R |= SYSCTL_RCC2_USERCC2;
            // 1) bypass PLL while initializing
            SYSCTL_RCC2_R |= SYSCTL_RCC2_BYPASS2;
            // 2) select the crystal value and oscillator source
            SYSCTL_RCC_R &= ~SYSCTL_RCC_XTAL_M;   // clear XTAL field
            SYSCTL_RCC_R += SYSCTL_RCC_XTAL_16MHZ;// configure for 16 MHz crystal
            SYSCTL_RCC2_R &= ~SYSCTL_RCC2_OSCSRC2_M;// clear oscillator source field
            SYSCTL_RCC2_R += SYSCTL_RCC2_OSCSRC2_MO;// configure for main oscillator source
            // 3) activate PLL by clearing PWRDN
            SYSCTL_RCC2_R &= ~SYSCTL_RCC2_PWRDN2;
            // 4) set the desired system divider and the system divider least significant bit
            SYSCTL_RCC2_R |= SYSCTL_RCC2_DIV400;  // use 400 MHz PLL
            SYSCTL_RCC2_R = (SYSCTL_RCC2_R&~0x1FC00000) // clear system clock divider field
            + (SYSDIV2<<22);      // configure for 80 MHz clock
            // 5) wait for the PLL to lock by polling PLLLRIS
            while((SYSCTL_RIS_R&SYSCTL_RIS_PLLLRIS)==0){};
            // 6) enable use of PLL by clearing BYPASS
            SYSCTL_RCC2_R &= ~SYSCTL_RCC2_BYPASS2;
          }
        \end{lstlisting}
        The 80MHz version sets \verb|SYSDIV2=4|, the 50MHz version
        sets \verb|SYSDIV2=7|, and the 16MHz version does not set the
        PLL at all.
      \item The interrupt is acknowledged by reading the \verb|COUNT|
        bit of the \verb|STCTRL| register.
      \end{enumerate}
    \end{solution}
  \end{Ex}
\item
  \begin{Ex}
    Look up the explicit sequence of events that occur as an interrupt
    is processed. Read section 2.5 in the TM4C123 data sheet
    (\verb|http://www.ti.com/lit/ds/symlink/tm4c123gh6pm.pdf|). Look
    at the assembly code generated for an interrupt service routine.
    \begin{enumerate}
    \item What is the first assembly instruction in the ISR? What is
      the last instruction?
    \item How does the system save the prior context as it switches
      threads when an interrupt is triggered?
    \item How does the system restore context as it switches back
      after executing the ISR?
    \end{enumerate}
    \begin{solution} \hfill \vspace{.75em} \\
      \begin{enumerate}
      \item The first action upon entry into the ISR is to put the
        \verb|EXC_RETURN| value into the \verb|LR|. The last
        instruction can be any of the three:
        \begin{itemize}
        \item A \verb|LDM| or \verb|POP|
        \item A \verb|BX|
        \item A \verb|LDR| with the \verb|PC| as the destination
        \end{itemize}
      \item The system saves context by pushing all the registers onto
        the stack.
      \item The system restores context by popping all the register
        values from the stack.
      \end{enumerate}
    \end{solution}
  \end{Ex}
\end{enumerate}
\section{Deliverables}
\begin{enumerate}
\item Drivers \\
  Since only one interrupt handler may be invoked by the NVIC per
  hardware interrupt, our RTOS (hereby viod) will have one driver per
  peripheral to handle the interrupts associated with external
  events. Drivers will allow threads to subscribe to each possible
  interrupt event, and when an interrupt is triggered the driver
  passes along the information contained in the interrupt to the task
  in the form of a \emph{notification}.
\item Timer \\
  Timers are capable of firing according to a specified
  frequency. With this feature we added a method
  \verb|timer_add_periodic_thread|.
\item ADC \\
  We envisioned that the ADC on our device would mainly be used for
  periodic data acquisition, so the driver was designed primarily with
  that use case in mind. We have designed the ADC driver to work in
  conjunction with our timer interface for quick and easy consumption
  of ADC data. When the user wishes to collect ADC data in a periodic
  manner, all the user needs to do is supply the frequency with which
  they would like to sample at, a buffer to be filled with analog
  data, and the timer to bind the sampling process to. When the timer
  is triggered, the ADC is immediately sampled and the supplied buffer
  is filled with new data.
\item UART \\
  The UART driver is capable of accepting susbcriptions from
  tasks. When an interrupt is triggered by an incoming character, the
  UART driver's ISR is invoked by the NVIC. This ISR acknowledges the
  interrupt and schedules a periodic task to trigger (once) before
  returning. This task will be responsible for calling the pseudo-ISR
  of each subscribed task with information about the incoming
  character.
\item LCD \\
  The LCD is capable of printing strings and characters. Two displays
  may be registered (in which case a horizontal line will separate the
  two) and text written to each independently.
\end{enumerate}
All specifications (objectives) above were met by the time of checkout
on Monday February 9.
\section{Hardware Design}
No hardware design is required for this lab.
\section{Software Design}
\begin{enumerate}[1)]
\item Low level LCD driver (\verb|ST7735_.c| and \verb|ST7735_.h|
  files) \\ \todo
\item Low level ADC driver (\verb|ADC.c| and \verb|ADC.h|) \\ \todo
\item Low level timer driver (\verb|OS.c| and \verb|OS.h|) \\ \todo
\item High level main program (the interpreter) \\ \todo
\end{enumerate}
\section{Measurement Data}
\begin{enumerate}[1)]
  \item Estimated time to run the periodic interrupt \\ \todo
  \item Measured time to run the periodic timer interrupt \\ \todo
\end{enumerate}
\end{document}
