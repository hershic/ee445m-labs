\documentclass[12pt]{article}
\title{EE445M Lab 1}
\author{Hershal Bhave (hb6279) and Eric Crosson (esc625)}
\date{Due Sometime Soon}

\usepackage[in]{fullpage}
\usepackage{cleveref}
\usepackage[nosolutionfiles]{answers}
\usepackage{xcolor}
\usepackage{enumerate}

\newenvironment{Ex}{\textbf{Problem}\vspace{.75em}\\}{}
\Newassociation{solution}{Soln}{Answers}
\pagebreak[3]
\newcommand{\Opentesthook}[2]{\Writetofile{#1}{\protect\section{#1: #2}}}
\renewcommand{\Solnlabel}[1]{\textbf{Solution}\quad}

\newcommand{\dd}[1]{\:\mathrm{d}{#1}}
\newcommand{\ddt}[1]{\frac{\dd{}}{\dd{#1}}}
\newcommand{\dddt}[1]{\frac{\dd{}^2}{\dd{#1}^2}}

\begin{document}
\maketitle
\section{Prepreparation}
\begin{enumerate}
\setcounter{enumi}{-1}
\item
  \begin{Ex}
    Please review the style guideline presented in style.pdf and
    \verb|c_and_h_files.pdf|. Go to the ARM site to download the
    compiler to your laptop
    \verb|https://www.keil.com/demo/eval/armv4.htm|. Do not get the
    newest compiler (Version 5.x) because it does not support the TI
    boards. Any version 4.7 to 4.73 will be OK. Download and install
    the Keil’s uVision4 compiler using these instructions
    \verb|http://www.ece.utexas.edu/~valvano/edX/KeilInstall.html|.
    \begin{solution} \hfill \vspace{.75em} \\
      Ok.
    \end{solution}
  \end{Ex}
\item 
  \begin{Ex}
    Search through the \verb|UARTInts_4C123| project to answer these
    questions about the UART port.
    \begin{enumerate}
    \item This example used UART0. What lines of C code define which
      port will be used for the UART channel? b) What lines of C code
      define the baud rate, parity, data bits and stop bits for the
      UART?
    \item Which port pins are used for the UART? Which pin transmits
      and which pin receives?
    \item Look in the uart.c driver to find what low-level C code
      inputs one byte from the UART port.
    \item Similarly, find the low-level C code that outputs one byte
      to the UART port.
    \item Find in the project the interrupt vector table. In
      particular, how does the system set the ISR vector?
    \item This code \verb|UART0_ICR_R = UART_ICR_TXIC;| acknowledges a
      serial transmit interrupt. Explain how the acknowledgement
      occurs in general for all devices and in specific for this
      device.
    \item Look in the data sheet of the TM4C123 and determine the
      extent of hardware buffering of the UART channel. For example,
      the 9S12 transmitter has a transmit data register and a transmit
      shift register. So, the software can output two bytes before
      having to wait. The serial ports on the PC have 16 bytes of
      buffering. So, the software can output 16 bytes before having to
      wait. The 9S12 receiver has a receive data register and a
      receive shift register. This means the software must read the
      received data within 10 bit times after the RDRF flag is set in
      order to prevent overrun. Is the TM4C123 like the 9S12 (allowing
      just two bytes), or is it like the PC (having a larger hardware
      fifo buffer)?
    \end{enumerate}
    \begin{solution} \hfill \vspace{.75em} \\
      \begin{enumerate}
      \item {\huge \color{red} TODO}
      \item {\huge \color{red} TODO}
      \item {\huge \color{red} TODO}
      \item {\huge \color{red} TODO}
      \item {\huge \color{red} TODO}
      \item {\huge \color{red} TODO}
      \item {\huge \color{red} TODO}
      \end{enumerate}
    \end{solution}
  \end{Ex}
\item
  \begin{Ex}
    Search through the \verb|ST7735\_4C123.zip| project to answer these
    questions about the LCD interface.
    \begin{enumerate}
    \item What synchronization method is used for the low-level
      command writedata?
    \item Explain the parameters of the function
      \verb|ST7735\_DrawChar|. I.e., how do you use this function?
    \item Which port pins are used for the LCD? Find the connection
      diagram needed to interface the LCD.
    \item Specify which other device shares pins with the LCD.
    \end{enumerate}
    \begin{solution} \hfill \vspace{.75em} \\
      \begin{enumerate}
      \item {\huge \color{red} TODO}
      \item {\huge \color{red} TODO}
      \item {\huge \color{red} TODO}
      \item {\huge \color{red} TODO}
      \end{enumerate}
    \end{solution} 
  \end{Ex}
\item
  \begin{Ex}
    Search through the \verb|PeriodicSysTickInts_4C123.zip|,
    \verb|ST7735_4C123|, and \verb|GPIO_4C123.zip| projects to answer
    these questions about the SysTick interrupts.
    \begin{enumerate}
    \item What C code defines the period of the SysTick
      interrupt?
    \item The \verb|GPIO\_4C123| project runs at 16 MHz, the
      \verb|PeriodicSysTickInts\_4C123| project runs at 50 MHz, and the
      \verb|ST7735\_4C123| project runs at 80 MHz. Find the RCC and RCC2
      registers in the data sheet. Look at these three projects to
      explain how the system clock is established. We will be running
      at 80 MHz for most labs in the class.
    \item Look up in the data sheet what condition causes this SysTick
      interrupt and how this interrupt is acknowledged?
    \end{enumerate}
    \begin{solution}
      \begin{enumerate} \hfill \vspace{.75em} \\
      \item {\huge \color{red} TODO}
      \item {\huge \color{red} TODO}
      \item {\huge \color{red} TODO}
      \end{enumerate}
    \end{solution}
  \end{Ex}
\item 
  \begin{Ex}
    Look up the explicit sequence of events that occur as an interrupt
    is processed. Read section 2.5 in the TM4C123 data sheet
    (\verb|http://www.ti.com/lit/ds/symlink/tm4c123gh6pm.pdf|). Look
    at the assembly code generated for an interrupt service routine.
    \begin{enumerate}
    \item What is the first assembly instruction in the ISR? What is
      the last instruction?
    \item How does the system save the prior context as it switches
      threads when an interrupt is triggered? c) How does the system
      restore context as it switches back after executing the ISR?
    \end{enumerate}
    \begin{solution} \hfill \vspace{.75em} \\
      \begin{enumerate}
      \item {\huge \color{red} TODO}
      \item {\huge \color{red} TODO}
      \end{enumerate}
    \end{solution}
  \end{Ex}
\end{enumerate}
\end{document}
