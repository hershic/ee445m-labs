\documentclass[12pt]{article}
\title{EE445M Lab 5}
\author{Hershal Bhave (hb6279) and Eric Crosson (esc625)}
\date{Friday April 10, 2015}

\usepackage[in]{fullpage}
\usepackage{listings}
\usepackage{tabularx}
\usepackage{cleveref}
\usepackage[nosolutionfiles]{answers}
\usepackage{graphicx}
\usepackage{xcolor}
\usepackage{color}
\usepackage{enumerate}
\usepackage{pdfpages}
\usepackage{float}
\usepackage{subcaption}

\newenvironment{Ex}{\textbf{Problem}\vspace{.25em}\\}{}
\Newassociation{solution}{Soln}{Answers}
\pagebreak[3]
\newcommand{\Opentesthook}[2]{\Writetofile{#1}{\protect\section{#1: #2}}}
\renewcommand{\Solnlabel}[1]{\textbf{Solution}\quad}
\newcommand{\todo}{\hfill{\LARGE \emph{\color{red}TODO}}}
\newcommand{\ohm}{$\Omega$}
\newcommand{\hbr}{\hfill\vspace{.25em}\\}
\newcommand{\dd}[1]{\:\mathrm{d}{#1}}
\newcommand{\ddt}[1]{\frac{\dd{}}{\dd{#1}}}
\newcommand{\dddt}[1]{\frac{\dd{}^2}{\dd{#1}^2}}

\definecolor{mygreen}{rgb}{0,0.6,0}
% \definecolor{mygreen}{rgb}{0.13,0.55,0.13}
\definecolor{mygray}{rgb}{0.5,0.5,0.5}
\definecolor{mymauve}{rgb}{0.58,0,0.82}

\lstset{
  backgroundcolor=\color{white},
  basicstyle=\scriptsize\ttfamily,
  breakatwhitespace=false,
  breaklines=true,
  captionpos=b,
  commentstyle=\color{mygreen},
  deletekeywords={...},
  escapeinside={\%*}{*)},
  extendedchars=true,
  frame=single,
  keywordstyle=\color{blue},
  rulecolor=\color{black},
  showspaces=false,
  showstringspaces=false,
  showtabs=false,
  stringstyle=\color{mymauve},
  tabsize=2,
  title=\lstname,
  columns=fullflexible,
}

\begin{document}
\maketitle

\section{Objectives}
\begin{itemize}

\item Interface a SD card to the TM4C123 SPI port.
\item Address translation from logical to physical address.
\item Write a file-system driver.
\item Implement a disk storage protocol.
\item Develop a simple directory system.
\item Stream debugging information onto the disk.
\item Add commands to your interpreter to test and evaluate the
  solid-state disk.
\end{itemize}

\section{Hardware Design}
None in this lab.

\section{Software Design}
Reference \cref{lst:lab5,lst:ff-h,lst:shell-h,lst:shell-c,lst:system-h,lst:system-c}.

\section{Measurement Data}
\begin{table}
  \centering
  \begin{tabular}{c|c}
    Read Bandwidth & Write Bandwidth \\
    \hline
    51 KB/s & \todo \\
  \end{tabular}
  \caption{SDCard Read/Write Bandwidth}
  \label{tbl:sdcard-bandwidth}
\end{table}

The SPI read/write bandwidth is shown in
\cref{tbl:sdcard-bandwidth}. The SPI clock rate is 8 MHz, confirmed by
oscilloscope and code. Two SPI packets can be seen in
\cref{fig:data-frames}.

\begin{figure}[H]
  \centering
  \begin{subfigure}[b]{0.45\textwidth}
    \includegraphics[width=\textwidth]{./img/TEK00001}
    \caption{A data read frame}
    \label{fig:data-frame-0}
  \end{subfigure}
  \begin{subfigure}[b]{0.45\textwidth}
    \includegraphics[width=\textwidth]{./img/TEK00002}
    \caption{Another data read frame}
    \label{fig:data-frame-1}
  \end{subfigure}
  \caption{Two SPI packets}
  \label{fig:data-frames}
\end{figure}

\section{Analysis and Discussion}
\begin{enumerate}
\item
  \begin{Ex}
    Does your implementation have external fragmentation? Explain with
    a one sentence answer.
    \begin{solution} \hbr
      Yes, during a write \verb|FatFs| iterates over the disk looking
      for free sectors, writing to each free sector as it is found
      until the disk is full or no more data exists to be written.
    \end{solution}
  \end{Ex}
\item
  \begin{Ex}
    If your disk has ten files, and the number of bytes in each file
    is a random number, what is the expected amount of wasted storage
    due to internal fragmentation? Explain with a one sentence answer.
    \begin{solution} \hbr
      \verb|FatFs| does not waste any space during any degree of
      fragmentation -- although files not might be contiguous, there
      are no wasted sectors in the write process. Fragmentation will
      not reduce the capacity of our disk.
    \end{solution}
  \end{Ex}
\item
  \begin{Ex}
    Assume you replaced the flash memory in the SD card with a high
    speed battery-backed RAM and kept all other hardware/software the
    same. What read/write bandwidth could you expect to achieve?
    Explain with a one sentence answer.
    \begin{solution} \hbr
      In this case the bottleneck of the system would be limited by
      the SPI bandwidth since RAM is much faster than SPI. SPI is not
      inherently limited so if this hypothetical new card reader was
      quick enough our bandwitdh limit would be the TM4C123GXL -- the
      devices could communicate as quickly as our main bus speed.
    \end{solution}
  \end{Ex}
\item
  \begin{Ex}
    How many files can you store on your disk? Briefly explain how you
    could increase this number (do not do it, just explain how it
    could have been done).
    \begin{solution} \hbr
      Given that we are using FAT32, we are able to store 4194304
      files. \\
      We can increase the maximum number of files that may be stored
      on the disk by shrinking the default cluster size -- smaller
      clusters will result in less wasted space for the last block of
      each file, allowing additional information to be packed into the
      same amount of storage space.
    \end{solution}
  \end{Ex}
\item
  \begin{Ex}
    Does your system allow for two threads to simultaneously stream
    debugging data onto one file? If yes, briefly explain how you
    handled the thread synchronization. If not, explain in detail how
    it could have been done. Do not do it, just give 4 or 5 sentences
    and some C code explaining how to handle the synchronization.
    \begin{solution} \hbr
    No. The first thread would have to take a semaphore before a write
    occurs and release it after the write goes through. The second
    thread would have to wait for the semaphore to be released before
    writing the data.
    \end{solution}
  \end{Ex}
\end{enumerate}

\newpage
\section{Code}
\lstinputlisting[language=C,label=lst:lab5,caption=\texttt{lab5.c}]{@doc-staging-area@/lab5.c}
\lstinputlisting[language=C,label=lst:ff-h,caption=\texttt{ff.h}]{@doc-staging-area@/ff.h}
\lstinputlisting[language=C,label=lst:shell-h,caption=\texttt{shell.h}]{@doc-staging-area@/shell.h}
\lstinputlisting[language=C,label=lst:shell-c,caption=\texttt{shell.c}]{@doc-staging-area@/shell.c}
\lstinputlisting[language=C,label=lst:system-h,caption=\texttt{system.h}]{@doc-staging-area@/system.h}
\lstinputlisting[language=C,label=lst:system-c,caption=\texttt{system.c}]{@doc-staging-area@/system.c}

\end{document}
