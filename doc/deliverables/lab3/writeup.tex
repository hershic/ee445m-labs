\documentclass[12pt]{article}
\title{EE445M Lab 3}
\author{Hershal Bhave (hb6279) and Eric Crosson (esc625)}
\date{Due Sometime Soon}

\usepackage[in]{fullpage}
\usepackage{listings}
\usepackage{cleveref}
\usepackage[nosolutionfiles]{answers}
\usepackage{graphicx}
\usepackage{xcolor}
\usepackage{color}
\usepackage{enumerate}

\newenvironment{Ex}{\textbf{Problem}\vspace{.75em}\\}{}
\Newassociation{solution}{Soln}{Answers}
\pagebreak[3]
\newcommand{\Opentesthook}[2]{\Writetofile{#1}{\protect\section{#1: #2}}}
\renewcommand{\Solnlabel}[1]{\textbf{Solution}\quad}
\newcommand{\todo}{{\LARGE \emph{\color{red}TODO}}}

\newcommand{\dd}[1]{\:\mathrm{d}{#1}}
\newcommand{\ddt}[1]{\frac{\dd{}}{\dd{#1}}}
\newcommand{\dddt}[1]{\frac{\dd{}^2}{\dd{#1}^2}}

\definecolor{mygreen}{rgb}{0,0.6,0}
% \definecolor{mygreen}{rgb}{0.13,0.55,0.13}
\definecolor{mygray}{rgb}{0.5,0.5,0.5}
\definecolor{mymauve}{rgb}{0.58,0,0.82}

\lstset{
  backgroundcolor=\color{white},
  basicstyle=\scriptsize\ttfamily,
  breakatwhitespace=false,
  breaklines=true,
  captionpos=b,
  commentstyle=\color{mygreen},
  deletekeywords={...},
  escapeinside={\%*}{*)},
  extendedchars=true,
  frame=single,
  keywordstyle=\color{blue},
  % language=Octave,
  % numbers=left,
  % numbersep=5pt,
  % numberstyle=\tiny\color{mygray},
  rulecolor=\color{black},
  showspaces=false,
  showstringspaces=false,
  showtabs=false,
  % stepnumber=2,
  stringstyle=\color{mymauve},
  tabsize=2,
  title=\lstname,
  columns=fullflexible,
}

\begin{document}
\maketitle

\section{Objectives}
\begin{enumerate}
\item \hfill {\huge \color{red} TODO}
\end{enumerate}

\section{Hardware Design}
No hardware design required for this lab.

\section{Software Design}
% - Printout of main program used to measure time jitter in Procedure 2
% - Printout of main program used test the blocking semaphores Preparation 4
% - Printout of your blocking/priority RTOS, os.c and any associated assembly files
\section{Measurement Data}
% - plot of the logic analyzer running the blocking/sleeping/killing/round-robin system (Lab 3.c)
% - plot of the scope window running the blocking/sleeping/killing/priority system (Lab 3.c)
% - table like Table 3.1 each showing performance measurements versus sizes of the Fifo and timeslices
\section{Analysis and Discussion}

% TODO: format

1) How would your implementation of \verb|OS_AddPeriodicThread| be different if there were 10 background
threads? (Preparation 1)
2) How would your implementation of blocking semaphores be different if there were 100 foreground threads?
(Preparation 4)
3) How would your implementation of the priority scheduler be different if there were 100 foreground threads?
(Preparation 5)
4) What happens to your OS if all the threads are blocked? If your OS would crash, describe exactly what the
OS does? What happens to your OS if all the threads are sleeping? If your OS would crash, describe exactly
what the OS does? If you answered crash to either or both, explain how you might change your OS to prevent
the crash.
5) What happens to your OS if one of the foreground threads returns? E.g., what if you added this foreground
void BadThread(void){ int i;
for(i=0; i<100; i++){};
return;
}
What should your OS have done in this case? Do not implement it, rather, with one sentence, say what the OS
should have done? Hint: I asked this question on an exam.
6) What are the advantages of spinlock semaphores over blocking semaphores? What are the advantages of
blocking semaphores over spinlock?
7) Consider the case where thread T1 interrupts thread T2, and we are experimentally verifying the system
operates without critical sections. Let n be the number of times T1 interrupts T2. Let m be the total number of
interruptible locations within T2. Assume the time during which T1 triggers is random with respect to the place
(between which two instructions of T2) it gets interrupted. In other words, there are m equally-likely places

\section{Code}
\lstinputlisting[language=C,label=lst:os-implementation,caption=\texttt{os.c}]{@doc-staging-area@/os.c}
\lstinputlisting[language=C,label=lst:os-header,caption=\texttt{os.h}]{@doc-staging-area@/os.h}
\lstinputlisting[language=C,label=lst:test-combination,caption=\texttt{test-combination.c}]{@doc-staging-area@/test-combination.c}

\end{document}
